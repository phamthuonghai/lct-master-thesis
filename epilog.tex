\chapter*{Conclusion}
\addcontentsline{toc}{chapter}{Conclusion}

In this thesis, we proposed several methods to exploit the sentence structure in NMT by manipulation of the attention mechanism.
While aiming at the same goal, the previous works focused mostly on the \seq2seq model utilizing RNNs, we explore the state-of-the-art Transformer model which was built entirely on the attention mechanism.
We speculate that this mechanism makes it easier for us to direct the neural network's behavior with additional syntactic information, e.g. the dependency tree.

We evaluated the models on Czech-to-English and German-to-Czech translations, both in relatively large data setting, against the \transformerbase (positional encoding) and \transformerrel (relative position).

The first set of methods attempted to enrich the encoder with source-side dependency tree.
First, we replaced the positional encoding and relative position with our proposed tree distance and tree traversal encoding.
BLEU scores of these models showed no improvement over the baselines.
However, combining our proposals with the baselines reported +0.5 to +0.8 BLEU against both baselines.
In addition, experiment results suggested that tree traversal works better than tree distance.

In this direction, we also proposed a specialized attention layer.
The difference between this and the standard attention layer is that the key and query come from an additional input sequence containing linguistic information, either POS tags or dependency labels.
The specialized POS head did not bring any improvement over baselines.
On the other hand, specialized dependency head brought +0.95 to +1.06 compared against the baselines without this modification.

We further invented a novel component of the Transformer model for sentence structure parsing by promoting the interpretation of self-attention as dependency syntax, and showed that the Transformer model can be used as a precise parser.
As suggested by the results, constraining self-attention to both true dependency as well as a simple diagonal matrix helped translation task at insignificant extra cost.
The best model of \DepParse and \DiagonalParse achieved +1.35 and +1.48 BLEU improvements, respectively. 
The models also performed well on parsing tasks.
\DepParse's best model achieved 91.56 UAS on the auto-generated treebanks, while \DiagonalParse, unsurprisingly, obtained 99.99\% accuracy.
Furthermore, \DepParse model performance is comparable to the referential parsers that were used to provide the parallel corpus with dependency trees.

It is clear from the results that enriching the \transformer with sentence structure can help.
However, the \transformer model is in fact able to capture this type of linguistic information already on its own and the guidance through multi-task learning is needed only as a small push towards trees following the annotation rules.


% Future works

\section*{Future Work}

While this thesis has explored various possibilities of exploiting sentence structure in NMT, we believe that there is still a vast room for improvements, which includes but is not limited to:

\paragraph{Fine-tuning the \DepParse model with gold-annotated data.}
Leveraging the self-attention weights to do both parsing and translation outperformed the baseline translation model and was comparable to referential parsers.
However, our model was trained on a synthetic treebank.
Hence, we could also try to fine-tune our model with the gold annotated treebank, which we believe should lead to a better parsing performance.

\paragraph{Examining dependency on subword units.}
In this thesis, we were working on the word level, without subword units, for an easier alignment with the dependency structure of the sentence.
Hence, all of the models had to face a serious out-of-vocabulary problem.
We would like to examine various methods to push the dependency relation beyond the word level, to subword level.
With that, we will be able to compare our proposed methods against the state-of-the-art translation models.
